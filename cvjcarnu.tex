\documentclass[10pt,a4paper]{moderncv}
\moderncvtheme[blue]{classic}                
\usepackage[utf8]{inputenc}
\usepackage[scale=0.85]{geometry}
% Je redéfinis les commandes, on ne sait jamais :)
\renewcommand*{\cvcomputer}[4]{%
  \cvdoubleitem{\textbf{#1}}{\small#2}{\textbf{#3}}{\small#4}}

\renewcommand*{\cvlanguage}[3]{%
  \cvline{\textbf{#1}}{\begin{minipage}[t]{.225\maincolumnwidth}#2\end{minipage}\hfill\begin{minipage}[t]{0.725\maincolumnwidth}\raggedleft\footnotesize\itshape #3\end{minipage}}}

%\setlength{\hintscolumnwidth}{3cm}
\AtBeginDocument{\recomputelengths}
\firstname{Jean-Christophe}
\familyname{Arnu}
\title{Ingénieur développeur et spécialiste PostgreSQL}              
\address{3 impasse du Petit Bois}{31130 Quint-Fonsegrives}    
\phone {05 61 54 72 63}            
\mobile{06 52 25 22 22}        
\email{jc.arnu@free.fr}                      
\extrainfo{38 ans, 2 enfants}
%\photo[64pt]{picture}     
%\homepage{www.francoistessier.info}
\nopagenumbers{}

\begin{document}
\maketitle
\section{Expériences professionnelles}
\subsection{Altidev}
	\cventry{Depuis Septembre 2012}{Ingénieur développement}{}{}{}{
		\begin{itemize}
			\item Editeur du logiciel Actiontracker : extension fonctionnelle et customization client,
			\item Continental : Afficheurs de production web, logiciel d'interface usine pour robots de découpe de circuits, système de gestion interventions sur lignes de production, outil d'aide à la gestion de production de prototypes,
			\item FibreExcellence : Logiciel de gestion du stock, logiciel d'achat de bois, logiciel de gestion des éléments sous pression,
			\item Expertise PostgreSQL : ministère de l'intérieur, EDF, CNES,
			\item UTC/UTAS : développement d'un outil d'affichage statistique de production et d'un afficheur de production de ligne de carénage de moteurs d'avions,
			\item Freescale Toulouse et Kuala Lumpur : divers outils mobiles,
			\item Maintien en conditions opréationnelles du système d'information interne.
		\end{itemize}
	}
\subsection{CS Système d'Information Toulouse}
              \cventry{Décembre~2008 -- Août 2012}{ Expert technique PostgreSQL et Logiciels libres}{niveau 3 (certification interne sur 4 niveaux)}{}{}{
                \begin{itemize}
                  \item Support interne aux projets de développement et avant-vente,
                  \item Veille et interaction avec la communauté,
                  \item Mise en œuvre de partenariats avec des sociétés spécialisées PostgreSQL.
              \end{itemize}}
              \cventry{écembre~2008 -- Août 2012}{Architecte du système mutualisé de production de logiciel du groupe CS}{}{}{}{
              \begin{itemize}
                \item {Définition, conception, exploitation et évolution de la plate-forme de production de logiciel du groupe CS (évolution architecturale et fonctionnelle/soft).}
                \item {Mise en œuvre d'un système tolérant aux pannes, maintien en condition opérationnelle et en condition de sécurité et évolution logicielle et architecturale afin de préserver les performances du système vital pour l'entreprise.}
              \end{itemize}
              }
              \cventry{Mars~2007 -- Mars~2008}{Ingénieur d'étude logiciel embarqué sur calculateur de vol FADEC/Turbomeca}{}{}{}{Développement d'un système d'exploitation temps réel pour un calculateur de vol d'une turbine d'hélicoptère. Logiciel écrit en C, utilisation de produits GNU (GCC) en cross compilation et création d'outils d'analyse de spécifications.}
\subsection{Paratronic Toulouse}
  \cventry{Mai~2001 -- Mars~2007}{Architecte système et développeur}{}{}{}
          {Conception et développement d'un système de collecte/supervision d'un réseau de télémesure de terrain pour la surveillance des cours d'eau et de la pluviométrie. Ce système avait pour vocation de collecter des informations télémétriques pour l'alerte du  personnel administratif d'astreinte en cas de phénomènes de crues.}
\subsection{IONIX Grenoble et Toulouse}
  \cventry{Juillet~2000 -- Mai~2001}{Responsable technique développement}{}{}{}
          { \begin{itemize}
              \item Mise en place d'un plan qualité pour l'ensemble des aspects techniques de l'entreprise,
              \item Gestion d'une équipe de 7 personnes dans le cadre de projets de développement (C++, PHP,...) et d'infogérance,
              \item Participation aux spécifications, conceptions et développements de projets de développement,
              \item Intervention avant vente et relation client. 
            \end{itemize}
          }
\subsection{IUX Toulouse}
  \cventry{Janvier -- Juillet~2000}{Ingénieur indépendant}{}{}{}
          {Travaux divers (Web (PHP), Développement en GPAO (Windows C++Builder), Formations (Linux : administration, logiciels, embarqué et drivers...)).
Clients : Motorola Toulouse, Union des Entreprises du Lot, Aérospatiale, Latécoère, ENAC, IB Formation, Banque Populaire, A.I.M.E.}          
\subsection{LAAS-CNRS Toulouse}
 \cventry{1996 -- 1999}{Stagiaire d'IUT, IUP, DEA et début de thèse}{}{}{}
          {Développement d'applications multimédia sous Sun (SunOS et Solaris) : application de transferts d'images JPEG, application de résultats de recherches sur des réseaux de Petri, développement d'une visioconférence synchronisée «Haut Débit». Focalisation sur la gestion de la qualité de service multimédia applicative et réseau.}

\section{Diplômes}
  \cventry{1999}{D.E.A Informatique Fondamentale et Parallélisme}
          {E.N.S.E.E.I.H.T.}
          {Toulouse}{}{}        
  \cventry{1998}{I.U.P Ingénierie des Systèmes Informatique}
          {Université Paul Sabatier (U.P.S.) Toulouse III}{}{}{}        
  \cventry{1996}{I.U.T Génie Électrique et Informatique Industrielle}
          {U.P.S. Toulouse III}{}{}
          {
%Option Automatismes et Informatique Industrielle
          }

\section{Formation professionnelle}
  \cventry{2010}{Certification pare-feu EdenWall IV}{EdenWall Paris}{Formation certifiante}{}{}{}
  \cventry{2010}{Savoir être des experts}{Sylvie Magonet Consulting}{Comportement et expression de l'expert dans et hors de l'entreprise}{}{}{}
  \cventry{2009}{Formation Python avancé}{Logilab}{Programmation avancée en langage Python}{}{}
  \cventry{2009}{Formation Talend}{Talend France}{Utilisation de l'ETL de Talend}{}{}
  \cventry{2008}{Formation Reqtify}{GeenSys Paris}{Formation complète sur l'outil de traçabilité Reqtify}{}{}

\section{Logiciel libre, communauté open-source}
  \cventry{Depuis 2003}{Implication dans la communauté PostgreSQL Francophone et Europe}{}{}{}{}
  \cventry{2008, 2010 et 2011}{PGDay Europe, PGConfEurope}{Communauté PostgreSQL Europe, Italie, Stuttgart, Amsterdam}{}{}{Conférence spécialisée autour de PostgreSQL, sa communauté et son éco-système.}
  \cventry{2008}{Organisation de la conférence PGDayFr}{Communauté Francophone PostgreSQL, PostgreSQLFr}{}{}{Une journée de conférence autour de PostgreSQL, PostGIS; présentation d'une mini-présentation sur la réplication de PostgreSQL par envoi de fichiers journaux.}
  \cventry{2008}{Rencontres Mondiales du Logiciel Libre}{Mont-de-Marsan}{}{}{Animation de deux tutoriels autour de PostgreSQL et de la réplication par envoi de fichiers journaux.}
  \cventry{2006}{Rencontres Mondiales du Logiciel Libre}{Dijon}{}{}{Présentation d'un cas d'utilisation de PostgreSQL pour une application critique.}

\section{Compétences}
  \subsection{Développement}
    \cvcomputer{Conception}{OMT, UML, SDL, HOOD.}{SGBD}{\emph{PostgreSQL}, Oracle, MS SQLServer, MySQL.}
    \cvcomputer{Langages}{Python, Java, C/C++, Delphi, Bash.}{Web}{Javascript, JQuery,XHTML/CSS, PHP.}
  
  \subsection{Système et réseau}
    \cvcomputer{OS}{Linux (Debian/RedHat), Unix, Windows.}{Plates-formes}{Apache HTTPD, Nginx, OpenSSH, mod\_security2.}
    \cvcomputer{Supervision}{RRDTool, Munin, Nagios.}{Sécurité}{IP/Netfilter, EdenWall.}
    \cvcomputer{Virtualisation, hypervision}{VirtualBox, VmWare (notion d'utilisation de la version server), connaissances de base sur Xen.}{Stockage}{SAN iSCSI et NAS.}
    \cvcomputer{Haute-disponibilité}{Réplication PostgreSQL, DRBD, réplication et synchronisation applicative, heartbeat, pacemaker.}{Réseaux}{Réseaux IPv4 et IPv6 avec gestion de la qualité de service, ATM (AAL3 à 5).}
  \subsection{Langues}
    \cvlanguage{Anglais}{lu, parlé, écrit.}{}
    \cvlanguage{Allemand}{notions.}{}
  
 \section{Centres d'intérêt}
  \cvline{Loisirs}{Astronomie, surf, badminton, VTT, ski, guitare, cuisine et lecture.}
  \cvline{Associatif}{Fondateur et ancien secrétaire et président de \mbox{PostgreSQLFr}\newline{}Membre de l'APRIL (association de défense et de promotion du logiciel libre) et de PostgreSQL Europe\newline{}Fondateur du Club des Utilisateurs de Linux de Toulouse et des environs (CULTe).\newline{}Participation au projet OpenStreeMap}
  \cvline{Autres}{Prévention et secours civiques de niveau 1}
\end{document}
